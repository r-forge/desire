\documentclass[a4paper, halfparskip+, DIV20, 10pt]{scrartcl}
\usepackage[latin1]{inputenc}
\usepackage[T1]{fontenc}
\usepackage{amsmath}
\usepackage[round]{natbib}
\usepackage{mathpazo}
 
\hyphenation{Tech-ni-sche}
\begin{document}
\title{\LARGE How to choose the appropriate Object-Oriented framework \\
  \LARGE Observations made while implementing \texttt{desiRe}}
\author{%
  \large Olaf Mersmann$^{\text{1}}$
  \and \large Heike Trautmann$^{\text{1}}$
  \and \large Detlef Steuer$^{\text{2}}$
  \and \large Claus Weihs$^{\text{1}}$
  \and \large Uwe Ligges$^{\text{1}}$
}
\date{}
\maketitle
\thispagestyle{empty}
\footnotetext[1]{Fakult�t Statistik, Technische Universit�t Dortmund}
\footnotetext[2]{Fakult�t WiSo, Helmut-Schmidt Universit�t Hamburg}
 
\vskip-3em The talk will focus on some of the challenges faced during the development of
the \texttt{desiRe} package. Desirability functions and desirability indices are powerful
tools for multicriteria optimization und multicriteria quality control purposes. The
package desiRe not only provides functions for computing desirability functions of
Harrington- \citep{harrington1965} and Derringer/Suich-type \citep{derringer1980} but also
allows the specification of the functions in an interactive way. Density and distribution
functions of the desirability functions and the desirabi\-lity index are integrated
including the possibility of random number generation \citep{steuer2005},
\citep{trautmann2006}. Optimization procedures for the desirability index and a method for
determining the uncertainty of the optimum influence factor levels \citep{trautmann2004a}
as wells as a control chart for the desirability index with analysis of out-of
control-signals are implemented \citep{trautmann2004}. The Desirability Pareto-Concept
allows focussing on relevant parts of the Pareto-front by integrating
a-priori-expert-knowledge in the multicriteria optimization process \citep{mehnen2007}.
 
One of the challenges faced when developing a new package is which implementation of
object orientation to use. The S language and its implementation \textsf{R} provide two
very different approaches. S3 methods are a straight forward implementation, where method
dispatch done based on the class of one of the arguments. This is both fast and easy to
implement. On the other hand, S4 provides a powerful method dispatch system and much nicer
data encapsulation with pre- and postconditions as well as a certain degree of type
safety. Some of the aspects we will cover in the talk include ease of implementation, the
speed of method invocation and how these systems deal with functions as first class
objects. Especially the last aspect will be dealt with, since function objects are a
natural way to represent a desirability.
 
The last part of the talk will focus on ways to improve the speed of critical functions by
rewriting them in \textsf{C(++)}.
 
\bibliography{useR_2008-abstract}{}
\bibliographystyle{plainnat}
\end{document}
 
%% (set-fill-column 90)
