\documentclass[a4paper, halfparskip+, DIV20, 10pt]{scrartcl}
\usepackage[latin1]{inputenc}
\usepackage[T1]{fontenc}
\usepackage{amsmath}
\usepackage[round]{natbib}
\usepackage{mathpazo}
 
\hyphenation{Tech-ni-sche}
\begin{document}
\title{\LARGE Multicriteria optimization using \textsf{R} \\
  and the desirability package \texttt{desire}}
\author{%
  \large Olaf Mersmann$^{\text{1}}$
  \and \large Heike Trautmann$^{\text{1}}$
  \and \large Detlef Steuer$^{\text{2}}$
  \and \large Claus Weihs$^{\text{1}}$
  \and \large Uwe Ligges$^{\text{1}}$
}
\date{}
\maketitle
\thispagestyle{empty}
\footnotetext[1]{Fakult�t Statistik, Technische Universit�t Dortmund}
\footnotetext[2]{Fakult�t WiSo, Helmut-Schmidt Universit�t Hamburg}
 
\vskip-3em Desirability functions and desirability indices are
powerful tools for multicriteria optimization und multicriteria
quality control purposes. The package \texttt{desire} not only
provides functions for computing desirability functions of Harrington-
\citep{harrington1965} and Derringer/Suich-type \citep{derringer1980}
but also allows the specification of functions in an interactive
manner. Density and distribution functions of the desirability
functions and the desirabi\-lity index are integrated including the
possibility of random number generation \citep{steuer2005},
\citep{trautmann2006}. Optimization procedures for the desirability
index and a method for determining the uncertainty of the optimum
influence factor levels \citep{trautmann2004a} as wells as a control
chart for the desirability index with analysis of out-of
control-signals are implemented \citep{trautmann2004}. The
Desirability Pareto-Concept allows focussing on relevant parts of the
Pareto-front by integrating a-priori-expert-knowledge in the
multicriteria optimization process \citep{mehnen2007}.

We will focus on the practical aspects of optimizing. First we will
give a short review of the traditional optimization strategy using
desirability indices. Then we will show an example of the procedure
using data from a chocolate production process \citep{alamprese2007}
an extend the optimization by including the variance of the estimates
in the optimization.
 
\bibliography{enbis_2009-abstract}{}
\bibliographystyle{plainnat}
\end{document}
 
%% (set-fill-column 90)
