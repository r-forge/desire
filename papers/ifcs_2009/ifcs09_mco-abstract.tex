\documentclass{svmult}

\usepackage{amsmath}
\usepackage{amsfonts}
\usepackage{latexsym}
\usepackage{graphicx}
\usepackage{multicol}
\usepackage{natbib}

\begin{document}

\title*{Multicriteria optimization in \textsf{R}}

\author{%
  Olaf Mersmann\inst{1}
  \and Heike Trautmann\inst{1}
  \and Claus Weihs\inst{1}}

\institute{Fakult�t Statistik, Technische Universit�t Dortmund}

\maketitle

\begin{abstract}
  Pareto concept based Multi-objective optimisation procedures have
  been a big research topic. Some of the notable solvers developed in
  this field are NSGA-II \citep{dep2002}, SPEA2 \citep{zitzler2001}
  and the SMS-EMOA \citep{beume2007}. So far their usage has been
  restricted to \textsf{C}/\textsf{C++} programs and to some extend
  \textsf{Matlab}. Our \textsf{R} package \texttt{mco} makes these
  routines, as well as auxiluary functions for the analysis and
  visualisation of the results obtained, vailable in the statistical
  programming environment \textsf{R}. Also included in the package is
  a large collection of test problems, with which one can compare and
  contrast the strengths and weaknesses of the different algorithms.

  %% Evtl noch erg�nzen woran Du im Moment arbeitest, wenn man das bis
  %% dahin implementiert bekommt.
  The talk will give a short introduction to \textsf{R} and then
  illustrate the capabilities of the \textsf{mco} package using
  several different classes of test problems.  

  \keywords{Muticriteria Optimization, R}
\end{abstract}

\bibliographystyle{plain}
\bibliography{ifcs09_olafm}
\end{document}
