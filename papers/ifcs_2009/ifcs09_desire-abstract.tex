\documentclass{svmult}

\usepackage{amsmath}
\usepackage{amsfonts}
\usepackage{latexsym}
\usepackage{graphicx}
\usepackage{multicol}
\usepackage{natbib}

\begin{document}

\title*{Desirabilitiy functions in multicriteria optimization \\
  Observations made while implementing \texttt{desiRe}}

\author{%
  Olaf Mersmann\inst{1}
  \and Heike Trautmann\inst{1}
  \and Detlef Steuer\inst{2}
  \and Claus Weihs\inst{1}
  \and Uwe Ligges\inst{1}}

\institute{%
  Fakult�t Statistik, Technische Universit�t Dortmund
  \and Fakult�t WiSo, Helmut-Schmidt Universit�t Hamburg}

\maketitle

\begin{abstract}
  Desirability functions and desirability indices are powerful tools
  for multicriteria optimization und multicriteria quality control
  purposes. The \textsf{R} package \texttt{desiRe} not only provides
  functions for evaluating desirability functions of the Harrington-
  \citep{harrington1965} and Derringer/Suich-type
  \citep{derringer1980} but also allows for their specification 
  in an interactive manner. Density and distribution functions of the
  desirability functions and the desirability index are integrated, as
  well as random number generators \citep{steuer2005},
  \citep{trautmann2006}. 
  
  The talk will illustrate the capabilities of the package by
  analysing a dataset first presented in \citet{alamprese2007}. Topics
  covered will include optimizing the desirability as well as
  its stochastic counterpart, the realistic desirability.

  \keywords{Desirability, R}
\end{abstract}

\bibliographystyle{plain}
\bibliography{ifcs09_olafm}
\end{document}
